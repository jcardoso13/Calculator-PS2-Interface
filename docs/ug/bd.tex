\section{Block Diagram}

The picoVersat block diagram is shown in Fig.~\ref{fig:bd}.

\begin{figure}[!htbp]
    \centerline{\includegraphics[width=\textwidth]{bd}}
    \vspace{0cm}\caption{Block Diagram}
    \label{fig:bd}
\end{figure}


PicoVersat contains 4 main registers: the accumulator (register A), the data
pointer (register B), the flags register (register C) and the program counter
(register PC).

\subsection{Accumulator register}

Register A, the accumulator, is the main register in this architecture. It can
be loaded with an immediate value from the instruction itself (immediate value)
or with a value read from the data interface. It is the destination of
operations using as operands register A itself and an immediate or addressed
value. Its value is driven out to the data interface.

\subsection{Pointer register}

Register B, the memory pointer, is used to implement indirect loads and
stores to/from the accumulator, respectively. Its contents is the
load/store address for the data interface. Register B itself is in the memory
map so it can be read or written as if accessing the data interface.

\subsection{Flags register}

Register C, the flags register, is used to store three operation flags: the
negative, overflow and carry flags. Register C itself is in the memory
map so it can be read or written as if accessing the data interface.

\subsection{PC register}

The Porgram Counter (PC) register contains the address of the next instruction to be fetched from the
Program Memory. The PC normally increments to fetch the next
instruction. In program jumps, the PC register is loaded with an
instruction immediate or with the register B value.

\subsection{Control Bus}
\label{sec:rwbus}


The Control Bus signals shown in Fig.~\ref{fig:bd} are described
in Table~\ref{tab:rwbus}.

\begin{table}[!htbp]
  \centering
    \begin{tabular}{|p{1.8cm}|c|p{10cm}|}
    \hline 
    {\bf Name} & {\bf Direction} & {\bf Description} \\
    \hline \hline 
     req & OUT & Read or write request.\\
    \hline
     rnw & OUT & Characterizes the request as read if it is 1 or a write if it is 0. \\
    \hline
     address & OUT & Address to be read or written \\
    \hline
     data2read & IN & Data to be read from the Control Bus \\
    \hline
     data2write & OUT & Data to be written to the Control Bus \\
    \hline

    \end{tabular}
  \caption{Control Bus signals as driven by the controller.}
  \label{tab:rwbus}
\end{table}
